\documentclass[11pt,a4paper]{article}

\usepackage[utf8]{inputenc}
\usepackage[czech]{babel}
\usepackage[unicode]{hyperref}
\usepackage{times}
\usepackage[left=2cm,top=3cm,text={17cm,24cm}]{geometry}

\begin{document}

\begin{titlepage}
    \begin{center}
        \textsc{\Huge Fakulta informačních technologii\\\vspace{0.4em}
        vysoké učení technické v Brně}\\
        \vspace{\stretch{0.382}}
        {\LARGE Typografie a publikování -- 4. projekt\\\vspace{0.3em}
        Bibliografické citace (BiBTeX, ČSN ISO 690)}
        \vspace{\stretch{0.618}}
    \end{center}
    \begin{flushleft}
        \today
        \hfill
        Pavel Yadlouski (xyadlo00)
    \end{flushleft}
\end{titlepage}

\newpage
\section*{Úvod}
	Dneska každý ví, co to je \uv{Microsoft Office} \footnote{Nějakou informaci můžete najít zde \cite{Microsoft2010}}. Podle analýzy 	Microsoft Office je nejpoužívanější software pro práci z textem. Ale asi většina z uživatelů Microsoft Office nikdy nevěděla, že existuje \LaTeX. V tom to projektu já bych chtěl vysvětlit co to vůbec je - \LaTeX, a jak on funguje

\section{Definice}
\LaTeX - je komplexní sada příkazů, využívající propracovaný sázecí program \TeX, pro přípravu široké
škály dokumentů, od vědeckých článků, zpráv, prezentací, až po celé knihy. \TeX\ i~samotný \LaTeX\ je
otevřený software. Jak otevřeni software, uživatel může opravovat \LaTeX podle svých potřeby a~svobodně šiřit, ale pod jiným jménem, aby nedošlo k~záměně. Pro pochopení rozdílu mezi \TeX a~\LaTeX můžete se podívat na \cite{Kopkac}.



\section{Práce s~{\LaTeX}em}
Pracující v \LaTeX vypadá jako programátor, protože jeho práce je skoro programování, ale ne programu, ale textového dokumentu. \uv{Programování} v \LaTeX sestává ze třech kroku, jak je uvedeno v~\cite{Rybicka}:
		\begin{enumerate}
			\item psaní (úprava) zdrojového textu,
			\item překlad\,--\,vysazení,
			\item prohlížení		
		\end{enumerate}

\section{Motivace}
Možná, už ve vaše hlavě zní otázka: proč by měl se učit nějaký skoro progromátorský jazyk, aby prostě sázet dokumenty? Odpověď je jednoduchá. Protože {\LaTeX} poskytuje spoustu balíků, který otevírají obrovské množství možností pro vytvořeni textů. 

V~článku \cite{Simecek} je uvedeno, vlastnosti {\LaTeX}u můžeme shrnout do několika slov: jednoduchost, elegance a~možnost být kreativní.

\section{Struktura dokumentu}
První, co má uživatel uvědomit při práci z~{\LaTeX}, je to, jak vypadá samostatný \uv{program} v~editoru. Kód se skládá ze dvou častí --  preambule a~vlastaního textu. Preambule obsahuje globální nastavení dokumentu (viz  \cite{Svamberg}).

\section{Prostředí pro práce}
Samozřejmě existuje spousta aplikaci pro práci v {\LaTeX}, který můžete naistalovat na svůj počítač, ale chtěl bych soustředit na online editory. Popis nějakých webových aplikací a~další informací můžete najít zde \cite{Sokol} 

\section{Co v \LaTeX je nejlepší}
Jeden z nejvýznamnějších atributů {\LaTeX} je sazba matematického textu. Výsledná kvalita matematických vzorců, tabulek, jakýchkoliv matematických symbolů je výrazně lepší v {\LaTeX}u, než v jiných textových editorech. Pro snadnou práci z~matematickým textem existuje specialně matematické prostředí. Takové prostředí začíná a~končí například pomocí \$..\$ nebo \$\$..\$\$. V~tomto módu lze velmi jednoduše sázet zlomky, matematické symboly a~znaky, závorky a~další speciálity viz \cite{Olsak}. Například matematický výraz může vypadat nejak takto :

\begin{eqnarray}
    \int_{b}^{a} x(x) dx & = & -\int\limits_{a}^{b} f(x) dx\\
     \overline{\overline{A \wedge B}} & \Leftrightarrow & \overline{\overline{A} \vee \overline{B}}
\end{eqnarray}

Příklad vysázené rovnice (lze nalézt v~\cite{CazarezCastro})
		
 


\newpage
	\bibliographystyle{czechiso}
	\renewcommand{\refname}{Literatura}
	\bibliography{proj4}
\end{document}
